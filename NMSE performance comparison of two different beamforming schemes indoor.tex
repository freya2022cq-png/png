import matplotlib.pyplot as plt
import numpy as np

# --- SCI 论文风格配置 ---
plt.rcParams.update({
    'font.family': 'serif',
    'font.serif': ['Times New Roman'],  # 确保系统安装了该字体,否则会回退到默认衬线体
    'font.size': 11,                    # 标准论文五号字左右
    'axes.labelsize': 12,
    'axes.titlesize': 12,
    'xtick.labelsize': 10,
    'ytick.labelsize': 10,
    'legend.fontsize': 10,
    'axes.linewidth': 1.2,              # 坐标轴线宽
    'grid.linewidth': 0.5,
    'lines.linewidth': 1.5,
    'lines.markersize': 7,
    'figure.figsize': (7, 5.5),         # 适合单栏排版的比例
    'figure.dpi': 300,                  # 高分辨率
    'savefig.dpi': 300,
    'savefig.format': 'pdf',            # 默认保存为矢量图 PDF
    'mathtext.fontset': 'stix',         # 使数学公式接近 Times
})

# --- 数据准备 (与最新 Web 版一致) ---
M = [256, 512, 1024]

# sigma^2 = 0.1
ideal_01 = [15.82, 15.82, 15.82]
dips_01  = [10.21, 11.53, 12.18]
dipi_01  = [11.58, 13.62, 14.25]

# sigma^2 = 1.0
ideal_10 = [12.45, 12.45, 12.45]
dips_10  = [6.78, 8.24, 8.62]
dipi_10  = [8.21, 10.58, 11.23]

# sigma^2 = 10.0
ideal_100 = [9.12, 9.12, 9.12]
dips_100  = [3.56, 4.92, 5.14]
dipi_100  = [4.91, 7.23, 7.89]

# --- 绘图 ---
fig, ax = plt.subplots()

# 颜色与标记定义
c_ideal = 'black'
c_dips  = '#004488' # 深蓝
c_dipi  = '#BB2222' # 深红

# (1) sigma^2 = 0.1: 实线 + 实心标记
ax.plot(M, ideal_01, color=c_ideal, linestyle='-', marker='o', label='Ideal ($\sigma^2=0.1$)')
ax.plot(M, dips_01,  color=c_dips,  linestyle='-', marker='s', label='DIP-S ($\sigma^2=0.1$)')
ax.plot(M, dipi_01,  color=c_dipi,  linestyle='-', marker='^', label='DIP-I ($\sigma^2=0.1$)')

# (2) sigma^2 = 1.0: 虚线 + 空心标记
ax.plot(M, ideal_10, color=c_ideal, linestyle='--', marker='o', markerfacecolor='white', label='Ideal ($\sigma^2=1.0$)')
ax.plot(M, dips_10,  color=c_dips,  linestyle='--', marker='s', markerfacecolor='white', label='DIP-S ($\sigma^2=1.0$)')
ax.plot(M, dipi_10,  color=c_dipi,  linestyle='--', marker='^', markerfacecolor='white', label='DIP-I ($\sigma^2=1.0$)')

# (3) sigma^2 = 10.0: 点线 + 异形标记
ax.plot(M, ideal_100, color=c_ideal, linestyle=':', marker='D', markersize=5, label='Ideal ($\sigma^2=10$)')
ax.plot(M, dips_100,  color=c_dips,  linestyle=':', marker='*', markersize=9, label='DIP-S ($\sigma^2=10$)')
ax.plot(M, dipi_100,  color=c_dipi,  linestyle=':', marker='x', markersize=8, label='DIP-I ($\sigma^2=10$)')

# --- 细节美化 ---
ax.set_xlabel('Feedback Bits ($M$)')
ax.set_ylabel('Average Sum Rate (bit/s/Hz)')
ax.set_title('(a) Indoor Scenario', fontweight='bold', pad=15)

# 设置 X 轴刻度
ax.set_xticks(M)
ax.set_xlim(200, 1100)
ax.set_ylim(2, 18)

# 网格与边框
ax.grid(True, linestyle='--', alpha=0.4)
# 强制显示四个边的边框 (Boxed Axes)
for spine in ax.spines.values():
    spine.set_visible(True)
    spine.set_edgecolor('black')

# 图例设置 (分为 3 列排放,避免遮挡曲线)
ax.legend(loc='upper center', bbox_to_anchor=(0.5, -0.15), 
          ncol=3, frameon=True, edgecolor='black', fancybox=False)

plt.tight_layout()

# --- 保存输出 ---
# 建议同时保存为 PDF (用于 LaTeX) 和 PNG (用于 Word/查阅)
plt.savefig('figure_indoor_scenario.pdf', bbox_inches='tight')
plt.savefig('figure_indoor_scenario.png', bbox_inches='tight', dpi=300)

print("SCI figures have been generated: figure_indoor_scenario.pdf and .png")
plt.show()