import numpy as np
import matplotlib.pyplot as plt
from mpl_toolkits.mplot3d import Axes3D
import matplotlib.colors as colors

def generate_subband_plot():
    # 1. 顶级学术期刊风格配置 (IEEE/Nature standard)
    plt.rcParams.update({
        "font.family": "serif",
        "font.serif": ["Times New Roman", "DejaVu Serif"],
        "font.size": 8,
        "axes.labelsize": 10,
        "axes.titlesize": 11,
        "xtick.labelsize": 9,
        "ytick.labelsize": 9,
        "mathtext.fontset": "stix",
        "axes.linewidth": 1.0
    })

    # 2. 公共参数设置
    subbands = np.arange(14)  # 子带范围 0-13
    samples = np.linspace(0, 10, 30)  
    X, Y = np.meshgrid(subbands, samples)
    
    # --- 数据逻辑 ---
    subband_data_a = {
        1: 810, 2: 210, 3: 15, 4: 10, 5: 8, 
        6: 5, 7: 5, 8: 5, 9: 10, 10: 15, 
        11: 160, 12: 480, 13: 720
    }
    Z_a = np.full(X.shape, 2.0) 
    for sb, val in subband_data_a.items():
        mask = (X == sb)
        Z_a[mask] = val + np.random.normal(0, 1.0, samples.shape)

    subband_data_b = {
        1: 450, 2: 130, 3: 45, 4: 35, 5: 30, 
        6: 30, 7: 35, 8: 40, 9: 60, 10: 110, 
        11: 340, 12: 460, 13: 550
    }
    Z_b = np.full(X.shape, 15.0) 
    for sb, val in subband_data_b.items():
        mask = (X == sb)
        Z_b[mask] = val + np.random.normal(0, 4.0, samples.shape)

    # --- 关键改进:统一颜色映射范围 ---
    # 获取两组数据的全局最小值和最大值,确保颜色条的完整性
    z_min = min(Z_a.min(), Z_b.min())
    z_max = max(Z_a.max(), Z_b.max())
    norm = colors.Normalize(vmin=z_min, vmax=z_max)

    # 3. 创建画布
    fig = plt.figure(figsize=(10, 5), dpi=300)

    # --- 绘制 (a) Indoor Scenario Subband-Energy ---
    ax1 = fig.add_subplot(121, projection='3d')
    # 使用 norm=norm 确保颜色映射一致
    surf1 = ax1.plot_surface(X, Y, Z_a, cmap='viridis', norm=norm,
                             linewidth=0.05, antialiased=True, alpha=0.9, 
                             edgecolor='w', rstride=1, cstride=1)
    ax1.set_title('(c) Indoor Scenario Subband-Energy', fontweight='bold', pad=3)
    
    # --- 绘制 (b) Outdoor Scenario Subband-Energy ---
    ax2 = fig.add_subplot(122, projection='3d')
    # 同样使用相同的 norm
    surf2 = ax2.plot_surface(X, Y, Z_b, cmap='viridis', norm=norm,
                             linewidth=0.05, antialiased=True, alpha=0.9, 
                             edgecolor='w', rstride=1, cstride=1)
    ax2.set_title('(d) Outdoor Scenario Subband-Energy', fontweight='bold', pad=3)

    # 4. 统一美化坐标轴
    for i, ax in enumerate([ax1, ax2]):
        ax.set_xlabel('Subband Index', labelpad=4)
        ax.set_ylabel('Dimension', labelpad=4)
        
        if i == 0:
            ax.set_zlabel('Energy (arb. units)', labelpad=6)
        
        ax.set_xlim(0, 13)
        ax.set_xticks(range(0, 14, 1))
        ax.set_zlim(0, 850)
        ax.set_zticks([0, 200, 400, 600, 800])
        
        ax.view_init(elev=30, azim=-60)
        ax.dist = 10.5 
        
        # 移除背景面颜色
        ax.xaxis.set_pane_color((1.0, 1.0, 1.0, 0.0))
        ax.yaxis.set_pane_color((1.0, 1.0, 1.0, 0.0))
        ax.zaxis.set_pane_color((1.0, 1.0, 1.0, 0.0))

    # 5. 颜色条与布局
    fig.tight_layout()
    plt.subplots_adjust(right=0.90, bottom=0.15, wspace=0.1, top=0.92)
    
    # 放置颜色条,现在它代表了全局的 z_min 到 z_max
    cbar_ax = fig.add_axes([0.92, 0.32, 0.012, 0.35])
    cbar = fig.colorbar(surf1, cax=cbar_ax) # surf1 和 surf2 现在使用相同的映射
    cbar.set_label('Energy Intensity', fontsize=9, labelpad=8)
    cbar.ax.tick_params(labelsize=8)
    cbar.outline.set_linewidth(0.8)

    plt.show()

if __name__ == "__main__":
    generate_subband_plot()